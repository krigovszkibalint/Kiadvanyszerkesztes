\documentclass[a4paper,12pt]{article}
\usepackage[T1]{fontenc}
\PassOptionsToPackage{defaults=hu-min}{magyar.ldf}
\usepackage[magyar]{babel}
\renewcommand{\theenumi}{\Roman{enumi}}
\renewcommand{\labelenumi}{\textbf{\theenumi.}}
\renewcommand{\theenumii}{\arabic{enumii}}
\renewcommand{\labelenumii}{\textit{(\theenumii)}}
\footnotestyle{rule=fourth}
\begin{document}
\begin{center}
	\textbf{Feladatok}
\end{center}
\begin{enumerate}
	\item Hány olyan tompaszögű háromszög van, melyben a szögek mérőszáma fokokban három különböző egész szám?
	\begin{enumerate}
		\item Adjon részletes indoklást!
		\item Szerkessze meg az egyik megoldást!
	\end{enumerate}
	\item Mi a feltétele annak, hogy egy derékszögű háromszög súlyvonalaiból, mint oldalakból szerkesztett háromszög derékszögű legyen?
	\begin{enumerate}
		\item Oldja meg úgy is a feladatot, ha az egyik oldalról tudjuk, hogy 3\,cm hosszú!
		\item Sorolja fel a felhasznált tételek neveit!
	\end{enumerate}
	\item Bizonyítsa be, hogy két ikerprímszám összege osztható 12-vel, ha a prímszámok 3-nál nagyobbak!
	\begin{enumerate}
		\item Írja le az ikerprímszám definícióját!\footnote{Lásd például Kiss`–-Mátyás könyvben.}
		\item Mit jelent a tökéletes szám fogalma?
	\end{enumerate}
\end{enumerate}
\begin{center}
	\textbf{Táblázat}
\end{center}
\begin{table}[!ht]
\centering
	\begin{tabular}{|l|r|r|r|r|}
	\multicolumn{1}{c}{}&\multicolumn{4}{c}{év} \\
	\cline{2-5}
	\multicolumn{1}{c|}{}&\multicolumn{1}{c}{\emph{2008}}&\multicolumn{1}{c}{\emph{2009}}&\multicolumn{1}{c}{\emph{2010}}&\multicolumn{1}{c|}{\emph{2011}}\\
	\hline
	\emph{jövedelem (Ft)}& 994\,000 & 1\,231\,500 & 1\,525\,410 & 2\,321\,600 \\
	\emph{járulék (Ft)}& 165\,000 & 194\,950 & 215\,750 & 235\,850 \\
	\hline
\end{tabular}
\caption{A jövedelem és járulékok kimutatása}
\label{table-jovedelem}
\end{table}
\Az{\ref{table-jovedelem}}.~táblázat\footnote{A táblázat számát kereszthivatkozásként írja be!} természetesen kitalált értékeket tartalmaz, csak a gyakorlás kedvéért kellenek.
\end{document}