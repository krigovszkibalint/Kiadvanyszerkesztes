\documentclass[12pt,twoside]{report}
\usepackage[T1]{fontenc}
\usepackage{fancyhdr}
\usepackage[unicode,bookmarksnumbered,colorlinks,linktocpage,allcolors=blue]{hyperref}
\usepackage[a4paper,top=40mm,bottom=40mm,inner=40mm,outer=30mm, headheight=15pt,headsep=8mm]{geometry}
\PassOptionsToPackage{defaults=hu-min}{magyar.ldf}
\usepackage[magyar]{babel}
\usepackage{hulipsum}
\pagestyle{fancy}
\fancyhf{}
\fancyhead[EL,OR]{\thepage}
\fancyhead[ER]{\nouppercase{\small\sffamily\leftmark}}
\fancyhead[OL]{\nouppercase{\small\sffamily\rightmark}}
\begin{document}
	\title{Példa a report dokumentumosztály
		használatára}
	\author{Szerző neve}
	\date{évszám}
	\maketitle
	\tableofcontents
\chapter*{Bevezetés}
\markboth{}{}
\hulipsum
\chapter{Fejezet címe}\label{fejezet-xy}
\section{Szakasz címe}
\hulipsum

Lásd még \cite{FAZEKAS}, illetve \cite{DAROCZY}[27.~oldal]. Ajánlott feladatgyűjtemények: \cite{DENKINGER,SOLT}.

Lásd \az{\ref{fejezet-xy}}.~fejezetet (\az{\pageref{fejezet-xy}}.~oldalon).
\begin{thebibliography}{4}
\bibitem{DAROCZY} Daróczy Zoltán: \emph{Mérték és integrál}, Budapest, 1984, Tankönyvkiadó.
\bibitem{DENKINGER} Denkinger Géza: \emph{Valószínűségszámítási gyakorlatok}, Budapest, 1986, Tankönyvkiadó.
\bibitem{FAZEKAS} Fazekas István: \emph{Valószínűségszámítás}, Debrecen, 2000, Debreceni Egyetem
Kossuth Egyetemi Kiadója.
\bibitem{SOLT} Solt György: \emph{Valószínűségszámítás}, Budapest, 1993, Műszaki Könyvkiadó.
\end{thebibliography}
\end{document}