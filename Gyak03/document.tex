\documentclass[a4paper,12pt]{article}
\usepackage[T1]{fontenc}
\PassOptionsToPackage{defaults=hu-min}{magyar.ldf}
\usepackage[magyar]{babel}
\usepackage{graphicx, url}
\footnotestyle{rule=fourth}
\begin{document}
\Az{\ref{figure-TeXszimb}}.~ábrán látható oroszlánt \textsc{Duane Bibby} –- neves amerikai grafikus –- tervezte. A kép innen letölthető:
\begin{center}
	\url{https://tomacstibor.uni-eszterhazy.hu/tananyagok/lion.pdf}
\end{center}
Ez az 5\,cm széles kép ma már a \LaTeX, pontosabban a \LaTeXe\ szimbólumává is vált.
\begin{figure}[!ht]
	\centering
		\includegraphics[width=5cm]{kacsa}
		\caption{A \TeX\ szimbóluma}
\label{figure-TeXszimb}
\end{figure}

\begin{figure}[!ht]
	\centering
	\includegraphics[width=5cm,angle=90,origin=c]{kacsa}%
	\includegraphics[width=5cm,angle=-90,origin=c]{kacsa}
	\caption{A \TeX\ szimbólum elforgatva}
	\label{figure-TeXszimb-forgatva}
\end{figure}
\Az{\pageref{figure-TeXszimb-forgatva}}.~oldalon látható \ref{figure-TeXszimb-forgatva}.~ábrát úgy kaptuk, hogy \az{\ref{figure-TeXszimb}}.~ábrát 90 illetve $-90$ fokkal elforgattuk a középpontja körül.\footnote{Az oldalon található hivatkozásokat automatikus kereszthivatkozásként illesszük a dokumentumba!}
\end{document}