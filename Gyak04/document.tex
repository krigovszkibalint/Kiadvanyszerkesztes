\documentclass[a4paper,12pt]{article}
\usepackage[T1]{fontenc}
\PassOptionsToPackage{defaults=hu-min}{magyar.ldf}
\usepackage[magyar]{babel}
\usepackage{graphicx,caption,url,multicol,xcolor}
\DeclareCaptionType{diag}[diagram]
\footnotestyle{rule=fourth}
\colorlet{sargaszold}{green!30!yellow}
\definecolor{macibarna}{RGB}{128,64,0}
\begin{document}
	\Az{\ref{fig-diagram}}.~diagramon\footnote{Forrás: \url{https://tomacstibor.uni-eszterhazy.hu/tananyagok/diagram.pdf}} az elmúlt év vizsgajegyeinek eloszlását láthatjuk.
	\begin{diag}[th!]
		\centering
		\includegraphics[width=10cm]{diagram}
		\caption{Tanulmányi eredmények 2011-ben}
		\label{fig-diagram}
	\end{diag}

Egy példa a dobozok használatára:

\begin{center}
	\fbox{\parbox{10cm}{Ez egy 10 cm széles doboz, ami kapott még egy keretet
is, majd középre helyeztük.
\par Ez a doboz természetesen csak a gyakorlás kedvéért készült, sok értelme nincs.}}
\end{center}

A következőkben egy kéthasábos szedést láthatunk.

\begin{multicols}{2}
	A hosszúság, terület, térfogat, ívhossz, felszín, egyszerű alakzatokra már az ókori görögök által definiáltak és számolhatóak voltak.
	
	A sokszögek területének és a poliéderek térfogatának fogalmát először \textsc{Peano} és \textsc{Jordan} terjesztették ki a
	sík illetve a tér részhalmazainak egy nagyobb rendszerére a XIX.~század	végén. Eszerint egy síkbeli korlátos halmaz külső mértéke legyen az őt lefedő véges sok sokszögből álló alakzatok területének pontos alsó korlátja, belső mértéke pedig a benne fekvő véges sok sokszögből álló alakzatok területének pontos felső korlátja.	Ha ezek egyenlőek, akkor a halmazt mérhetőnek, ezen közös értéket pedig a halmaz mértékének nevezzük. Térfogat esetén hasonló az eljárás.
	
	Ez a mértékfogalom egyszerű, de	az integrálás céljára nem megfelelő. Az általánosítás területén a fő lépést \textsc{Lebesgue} tette meg a XX.~század elején. Az általa alkotott mérték és integrál előnye a nagyobb általánosság, az integrál és a határátmenet felcserélhetősége.
\end{multicols}
\color{blue}
{30\.\% zöldhöz 70\,\% sárgát keverve a következő színt kapjuk: {\color{green!30!yellow}\rule{8mm}{3mm}}Az előző színt kereszteljük \texttt{sargaszold} névre: {\color{sargaszold}\rule{8mm}{3mm}} Definiáljon \texttt{macibarna} nevű színt, melynek RGB kódja 128, 64, 0: {\color{macibarna}\rule{8mm}{3mm}}
\end{document}