\documentclass[a4paper,12pt]{article}
\usepackage[T1]{fontenc}
\PassOptionsToPackage{defaults=hu-min}{magyar.ldf}
\usepackage[magyar]{babel}
\usepackage{hulipsum,amsthm,amssymb,mathtools}
%\renewcommand{\qedsymbol}{$\blacksquare$}
\newtheorem{tetel}{Tétel}[section]
\newtheorem{lemma}[tetel]{Lemma}
\theoremstyle{definition}
\newtheorem{defin}[tetel]{Definíció}
\theoremstyle{remark}
\newtheorem{megj}[tetel]{Megjegyzés}
\begin{document}
	\title{Cikk címe}
	\author{Szerző neve}
	\date{2021. augusztus 12.}
	\maketitle
\begin{abstract}
	\hulipsum[1]
	\section{Szakasz címe}
	\subsection{Alszakasz címe}
	\hulipsum
	\begin{tetel}
		Tétel szövege
	\end{tetel}
	\begin{proof}
		Bizonyítás szövege
	\end{proof}
	\begin{defin}
		Definíció szövege
	\end{defin}
	\begin{tetel}[Pitagorasz]\label{tetel-Pitagorasz}
		Tétel szövege
	\end{tetel}
	\begin{lemma}
		Lemma szövege
	\end{lemma}
	\begin{proof}[\Az{\ref{tetel-Pitagorasz}}.~tétel bizonyítása]
		Bizonyítás szövege
		\begin{equation*}
			a^2+b^2=c^2 \qedhere
		\end{equation*}
	\end{proof}
	\begin{megj}
		Megjegyzés szövege
	\end{megj}
	\paragraph{Paragrafus címe} Paragrafus szövege
\end{abstract}
\end{document}